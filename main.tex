% File: main.tex
\documentclass{kennsluakademia_conf}

\address{Veröld Háskóla Íslands}
\date{22 nóvember, 2024}
\keywords{Lykilorð 1, lykilorð 2, lykilorð 3, lykilorð 4}

\title{Titill}

\author{%
Fyrsti höfundur\autid{1}{0000-0000-0000-0000}, 
Annar höfundur\autid{1}{0000-0000-0000-0000}, 
Þriðji höfundur\autid{2}{0000-0000-0000-0000}%
}

% Define affiliations
\affil{1}{Deild, háskóli}
\affil{2}{Önnur deild, háskóli}

\begin{document}

\maketitle

\section{Inngangur}

Inngangur lýsir vandamáli eða áskorun sem höfundur/ar stóðu frammi fyrir í kennslu og markmiði verkefnisins.  

Kennsluakademían leggur áherslu á að styðja bæði rannsóknarverkefni og kennsluþróunarverkefni/umbætur á kennslu sem miða að því að efla gæði náms og kennslu. Akademían samþykkir  útdrætti sem byggir á rannsóknum tengdum háskólum, rannsóknarverkefnum á eigin kennslu (SotL), sem og útdrætti sem lýsa umbótum á kennsluaðferðum og kennsluþróun sem miða að því að innleiða nýjar aðferðir, verkfæri eða stefnu til að bæta námsupplifun nemenda og kennslugæði. 

Með þessu vill Kennsluakademían skapa vettvang þar sem fram fer umræða um hvoru tveggja fræðilegar rannsóknir á kennslu og hagnýtar kennsluumbætur.  Höfundar hvattir til að setja verkefni í fræðilegt samhengi, með því að vísa í rannsóknir á tengdum sambærilegum verkefnum.  

\section{Aðferð}

Aðferð lýsir framkvæmd verkefnisins, þeirri aðferð eða inngripi sem var notuð til að leysa vandamálið og hvaða gögnum var safnað til að meta árangur verkefnisins. 

\section{Niðurstöður}

Niðurstöður lýsa árangri af verkefninu, t.d. hvaða áhrif hafði verkefnið á náms árangur nemenda og upplifun af náminu.

\section{Umræður}

Í umræðum eru niðurstöður ræddar svo sem hvað hefði mátt betur fara og hvað mögulegu úrbætur væri hægt að gera í framtíðinni. Í umræðum eru niðurstöður jafnfram ræddar út frá rannsóknium annar á viðfangsefninu.  

\nocite{*}

\bibliographystyle{plainnat}  
\bibliography{references}  

\end{document}