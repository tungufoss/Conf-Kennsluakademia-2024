\documentclass[
    NAME={Dr. Helga Ingimundardóttir},
    EMAIL={helgaingim@hi.is},
    FACULTY={Iðnaðarverkfræði},
    TITLE={Endurskoðun á námskeiði í Viðskiptagreind: },
    SUBTITLE={Hagnýt hæfni í brennidepli},
    SEMINAR={Ráðstefna kennsluakademíunnar},
    DATE={22 nóvember, 2024},
    WIDE={true},
    ICELANDIC={true}
]{HI-LaTeX/hi-beamer}
\usepackage{multimedia}
\usepackage{tikz}
\usetikzlibrary{positioning, arrows, shapes, fit}
\usepackage{textcomp}

\begin{document}

% Section: Inngangur
\section{Inngangur}
\begin{frame}{Inngangur}
    \begin{itemize}
        \item Misræmi milli hæfni nemenda og atvinnulífs
        \item Endurgjöf frá útskrifuðum nemendum:
        \begin{itemize}
            \item Áhersla á \textit{Learning-by-Doing} (LbD)
            \item Betri notkun á \textit{GitHub} fyrir verkstjórn og samvinnu
        \end{itemize}
        \item Innsýn úr vinnustofu LOUIS verkefnisins:
        \begin{itemize}
            \item Endurskoðun byggð á \textit{Understanding by Design} (UbD)
        \end{itemize}
    \end{itemize}
\end{frame}

% Section: Aðferð
\section{Aðferð}
\begin{frame}{Aðferðafræði}
    \begin{itemize}
        \item Verkefnamiðuð námsaðferð (\textit{Problem-Based Learning}, PBL)
        \item Þróun \textit{Capstone}-verkefna:
        \begin{itemize}
            \item Samþætting hagnýtrar notkunar við raunverulegar áskoranir
            \item Áhersla á gagnvirka og rökstudda hugsun
        \end{itemize}
        \item Uppbygging námskeiðs:
        \begin{itemize}
            \item Lotur: gagnasiðfræði, mállíkön, klösun og ferlagreining
            \item Lokaverkefni sem samþættir öll viðfangsefni
        \end{itemize}
    \end{itemize}
\end{frame}

% Section: Niðurstöður
\section{Niðurstöður}
\begin{frame}{Niðurstöður}
    \begin{itemize}
        \item Betri þátttaka nemenda:
        \begin{itemize}
            \item Nær fullkomin mæting, jafnvel án mætingarskyldu
        \end{itemize}
        \item Endurgjöf nemenda:
        \begin{itemize}
            \item Nám krefjandi en lærdómsríkt
            \item Betri skilningur á raunverulegum verkefnum
        \end{itemize}
        \item Samantektarverkefni:
        \begin{itemize}
            \item Verkefni tengd atvinnulífi sem stuðla að hagnýtri reynslu
        \end{itemize}
    \end{itemize}
\end{frame}

% Section: Umræður
\section{Umræður}
\begin{frame}{Umræður og næstu skref}
    \begin{itemize}
        \item Áhersla á hópmyndun og traust:
        \begin{itemize}
            \item Þróun nýrrar lotu í upphafi námskeiðs
        \end{itemize}
        \item Endurbætur í námsmati:
        \begin{itemize}
            \item Bætt við hlekkjum að örnámskeiðum
        \end{itemize}
        \item Áhrif á atvinnulífið:
        \begin{itemize}
            \item Fyrirtæki nýta niðurstöður nemenda
        \end{itemize}
    \end{itemize}
\end{frame}

% Thank You Slide
\begin{frame}
    \centering
    \Huge Takk fyrir! \\
    \normalsize Spurningar?
\end{frame}


\end{document}
